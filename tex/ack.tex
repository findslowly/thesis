%%==================================================
%% thanks.tex for SJTU Master Thesis
%% based on CASthesis
%% modified by wei.jianwen@gmail.com
%% version: 0.3a
%% Encoding: UTF-8
%% last update: Dec 5th, 2010
%%==================================================

\begin{thanks}
First, I would like to acknowledge my advisor, Professor Kenny Zhu. 
This work cannot be done without him. 
He has provide invaluable suggestions and advices in the whole process of conducting this project. 
I want to especially thank for Kenny's time and energy.
Although Kenny is really busy, he always tries to have a personal meeting with us to keep us boosted and give us insights.  
In the experiment, we came across many failures and disatisfaction, sometimes for the low performance of our system, and sometimes for the inaccuracy of our data. 
But Kenny tries to pull us through, and act as the man who constantly tell us keep trying. 
Kenny is a really nice friend and a excellent professor, it is like start a voyage with him for walking through this project. 
Sometimes I feel like I am not giving all my heart into it, just like doze off when you are driving a ship. 
But whenever we are having a meeting, Kenny would shake me and help me correct the directions. 
Of course, the ship we drive, the project we have done does not get the perfect destination, there is still much room for improvement. 
But I am very happy about this research experience, it opened my eye for what research may looks like. 
It has difficulties that can discourage you, hardship that can intimidate you, but above all, it has the ecstasy when you found something new, something unexpected all the way through.  

Then Guoteng Rao and Biman Tang also deserve my acknowledgement. 
Although they are already graduated, they provide us with their previous work which functions as the base of this project. 
Especially for the data collection part, they must have spent so much in downloading those audio clips and arrange them into good form. 
Issac Newton once said, ``If I have seen further, it is by standing on the shoulders of giants''. 
I think Guoteng and Biman provides the shoulders for us to step on, so thank you, Guoteng and Biman. 

Finally, I would give my acknowledgement to my research mate and good friend, Xinyu Hua. 
We have done this project together for a long time, he is the good companion whenever I want to discuss some doubts or question. 
During all the working process, we held meeting with Kenny together and consult each other for the problems. 
Although we have a weekly meeting, I still meet with Xinyu a lot for the discussion of our project. 
Often, in the late night, we would chat on Skype to talk about the progress of this project and where we could try next. 
These small discussion with Xinyu always keep me energized. 
Since Xinyu has done the most job on the text analysis in this project, I really want to thank him for all his hardwork. 
He provided with me not only a scene-event map to build our audio scene recognition system, but a friendship that I will treasure even after this project. 

Thanks for all the friends in our ADAPT lab, and all the research seminars Kenny held for stimulating our interest and progress.   
 

\end{thanks}
