\chapter{Scene Recognition}

\section{Scene-Event Relation Mining}

\section{Audio Segmentation}
Once the models are built, we are ready to use it to test on a potential event clip. 
But for the problem of scene recognition, we need to detect the   events happening in a clip, therefore, a audio segmentation technique is employed to segment audio clips. \\ 

There are lots of research carried out on the segmentation of audios.
Some may simply use the average frame energy as the threshold to cut the clips, while other complicated methods involve more features and even use machine learning techniques. 
In this project, we use a method of setting a threshold combining frame energy and spectral centroids, and this method is described in \cite{giannakopoulos2009method}.
This method uses two features from audio:
\begin{itemize}
\item{Frame Energy: this features depicts the overall energy level of one frame. Typically frames where there are some events happening have a higher energy than silent frames.}
\item{Spectral centroid: this represents the "center of gravity" of the audio spectrum. For example, the spectral centroid of the i-th frame (denoted by $ C_i$) is given by the equation:
\[
C_i = \frac{\sum\limits_{k=1}^Nk\times Amp(k)}{\sum\limits_{k=1}^NAmp(k)}
\]
}
\end{itemize}

In this equation, $Amp(k)$ is the amplitude corresponding to bin $k$ in Discrete Fourier Transform (DFT) spectrum. 
So a higher spectral centroid denotes a higher frequency in this frame. Because noises are often in a low frequency, we could use this feature to filter them out. 

\section{Bayesian Network for Inference}

\section{Summary}
