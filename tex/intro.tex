%%==================================================
%% chapter01.tex for SJTU Master Thesis
%%==================================================

\chapter{Introduction}

\section{Motivation and Problem Description}
Audio data often contains many information that can be captured. For example, in an audio recorded under the context of office, we may here "computer keyboard typing" and "phone ringing", sometimes with "people talking" as the background sound. The detection of these information is important for audio tagging, or audio context inference. With the spread of mobile devices, we have a convinient access to audio streams. So an automatical technique for analyzing these data is useful for extracting information in audio and improving functionality in mobile devices. \\ 
In this thesis, we refer to the overall atmosphere or context of an audio clips as "scene". It is usually characterised by the name of a place, e.g., street, office, restaurant, stadium, etc. 

\section{Our Approach}
So in this thesis, we proposed a system of automatically detect the events in an audio clip and then infer the scene or context from the detected audio events. 

\section{Thesis Organization}
The thesis is organizaed as follows: 

