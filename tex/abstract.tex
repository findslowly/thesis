
\begin{englishabstract}
Audio scene recognition is the process of recognize the scene where a audio clip is recorded. 
Previously, researchers often trained model directly on scenes. 
In this work, we first train models on audible events, which are shorter sound like car honk, phone ringing. 
Then we infer the scene in a audio clip by the detected events in it. 
Gaussian Mixture Models are used as the training models and Mel-Frequency Cepstrum Coefficients (MFCCs) are used as the main features. 
We have done evaluation both on the event detection and scene recognition. 
The F-measure of our model on event detection is 0.4864, while the scene recognition reaches to an accuracy of 54$\%$ in a 10 scene classification task.  \\  

\englishkeywords{\large Audio Scene Recognition, Gaussian Mixture Model, Audio Event Detection}
\end{englishabstract}

\begin{abstract}
音频场景识别是判断一个音频所录制的场景的问题。
在之前的研究中,研究者们大多从场景上直接建立模型去检测新的音频。
本文的方法与前人不同的是采用直接对声音事件建立模型。
声音事件是比音频场景更加具体的声音,比如喇叭,电话声。
我们对音频场景分类的方法是将其切分,然后检测每个小段是属于什么声音事件。
我们然后根据检测到的声音事件来推测原来整个音频是属于什么场景的。 
在这过程中,我们利用高斯混合模型对声音事件建模,并且对剧本中的数据进行分析,从而得到场景与事件的统计关系。
在我们的测试中,我们对16个声音事件的分类F值达到0.4862,在10个音频场景分类中的准确率达到54$\%$。\\ 

\keywords{\large 音频场景分类 \quad  高斯混合模型\quad 声音事件检测}
\end{abstract}
